%!TEX program = xelatex
\documentclass[a4paper]{exam}
\usepackage{ctex}
\usepackage{CJKnumb}
\newif\ifprint
%\printtrue %选择题答案/printfalse时打印,/printtrue时隐藏
\printfalse
\usepackage{ulem}
\newcommand{\blank}[2][1cm]{\uline{\makebox[#1][c]{%
			\ifprint
			\phantom{#2}%
			\else
			#2%
			\fi}}}
\pagestyle{headandfoot}

\begin{document}
\section{选择题}
\begin{questions}
	\question 冯·诺依曼计算机中指令和数据均以二进制形式存放在存储器中,CPU区分它们的依据是\blank[0.5cm]{}
	\begin{choices}
		\choice 指令操作码的译码结果
		\choice 指令和数据的寻址方式
		\choice 指令周期的不同阶段
		\choice 指令和数据所在的存储单元
	\end{choices}

	\question 以下关于计算机历史发展的叙述,错误的是\blank{}
	\begin{choices}
		\choice 第一台电子计算机采用的是电子管技术
		\choice 早期的电子计算机,主要用于个人业务处理
		\choice 集成电路技术的发展,使电子计算机在体积和速度上有了很大改善
		\choice 摩尔定律说明了半导体集成度发展的规律
	\end{choices}

	\question 以下哪一项,不属于CPU的组成部分\blank{}
	\begin{choices}
		\choice 运算器
		\choice 控制器
		\choice 寄存器
		\choice I/O系统
	\end{choices}

	\question 下列关于总线仲裁方式的说法中,错误的为\blank{}
	\begin{choices}
		\choice 独立请求方式响应时间最快,是以增加处理机开销和增加控制线数为代价的
		\choice 计数器定时查询方式下,有一根总线请求(BR)和一根设备地址线,计数器可以从0开始增加,或接上次计数增加
		\choice 链式查询方式对电路故障最敏感
		\choice 分布式仲裁控制逻辑分散在各总线各部件中,不需要中央仲裁器
	\end{choices}

	\question 冯·诺依曼机可以区分指令和数据的部件是\blank{}
	\begin{choices}
		\choice 总线
		\choice 控制器
		\choice 控制存储器
		\choice 运算器
	\end{choices}

	\question 计算机的Cache——主存层次,主要是为了解决什么问题\blank{}
	\begin{choices}
		\choice 速度匹配问题
		\choice 存储器容量问题
		\choice 数据格式兼容问题
		\choice 电平匹配问题
	\end{choices}

	\question 总线的特征不包括以下哪一项\blank{}
	\begin{choices}
		\choice 物理特征
		\choice 功能特征
		\choice 电气特征
		\choice 时间特征
	\end{choices}

	\question 设CPU与I/O设备以中断方式进行数据传送,CPU响应中断时,该I/O设备接口控制器发送给CPU的指令中断向量表(中断向量表中放中断向量)的指针是\verb|0008H|。\verb|0008H|单元中的值是\verb|1200H|。则该I/O设备的中断服务程序在内存中的入口地址为\blank{}
	\begin{choices}
		\choice \verb|0008H|
		\choice \verb|0009H|
		\choice \verb|1200H|
        \choice \verb|1201H|
	\end{choices}
    
    \question DMA方式的接口电路中有程序中断部件,其作用为\blank{}
    \begin{choices}
    	\choice 进行预处理
    	\choice 向CPU提出总线使用权
    	\choice 向CPU提出传输结束
        \choice 检查数据是否出错
    \end{choices}

    \question 设浮点数的基数为8,尾数用原码表示,则以下\blank{}是规格化小数
    \begin{choices}
    	\choice 1.000101
    	\choice 0.000101
    	\choice 1.011011
    	\choice 0.000010
    \end{choices}

    \question 以下关于加法器的说法,正确的是\blank{}
    \begin{choices}
    	\choice 多个半加器串联可以实现两个多位二进制数的加法
    	\choice 一位全加器有两个输入端,分别为被加数和加数
    	\choice 全加器的输入端只要有1,和即为1
    	\choice 超前进位加法器可以快速计算出各位计算所需要的进位
    \end{choices}

    \question 在微程序控制方式中,以下说法正确的是\blank{}
    \begin{choices}
    	\choice 采用微程序控制器的处理器称为微处理器
    	\choice 每一条水平微指令发出一个或多个微操作指令
    	\choice 在微指令编码中,执行效率最低的是直接编码方式
    	\choice 垂直型微指令能充分利用数据通路的并行结构
    \end{choices}

    \question 下列关于RISC的叙述中,错误的是\blank{}
    \begin{choices}
    	\choice RISC普遍采用微程序控制器
    	\choice RISC处理器的单挑指令执行速度较快
    	\choice RISC的内部通用寄存器数量相对CISC多
    	\choice RISC的指令数寻址方式和指令格式相对CISC少
    \end{choices}

    \question 某计算机的指令流水线由四个功能段组成,指令流经各功能段的时间(忽略各功能段之间的缓存时间)分别为90ns,80ns,70ns和60ns。则该计算机的CPU时钟周期至少是\blank{}
    \begin{choices}
    	\choice 70ns
    	\choice 80ns
    	\choice 90ns
    	\choice 60ns
    \end{choices}

    \question 在一个时钟周期内将一个功能部件使用多次,这种技术称为\blank{}
    \begin{choices}
    	\choice 超标量技术
    	\choice 超长指令技术
    	\choice 超流水技术
    	\choice 超数据流
    \end{choices}
\end{questions}

\section{填空题}
\begin{questions}
	\question 在规格化浮点数中,如果阶码取5位,含一个符号位,尾数取6位。含一个符号位,试问其能表示的最小正数的规格化形式是\blank{}
	\question 设有机器数10101111,此数如为整数的反码表现形式,包含一个符号位,请问其真值的十进制形式为\blank{}
	\question 假设磁盘存储器共有6个盘片,最外侧两个记录用不使用,每面有204个磁道,没条磁道有12个扇区,每个扇区有512B容量,请问该磁盘存储器的总容量是\blank{}MB
	\question 某机器指令长度为16位,其中操作码字段和地址码字段均为4位,假设其有15条3地址指令,15条2地址指令,15条1地址指令,那么还可以设计\blank{}条0地址指令
	\question 符号相同的两个定点数相加,结果符号与原操作数的符号\blank{}则溢出
	\question 一般来说,指令周期分为取指周期、间指周期、\blank{}和中断周期四个阶段
	\question 某计算机系统中软盘驱动器以中断方式与处理机进行I/O通信,通信以16bit为传输单元,传输率为50kB/s,每次输送的开销(包括中断)为100个节拍,处理器的主频为50MHz,则磁盘使用时占用处理器时间的比例为\blank{}
	\question 动态存储器的刷新方式包括\blank{},分散刷新和异步刷新。
	\question 假设某系统总线在一个总线周期中并行传输4B信息,一个总线周期占用2个时钟周期,总线时钟频率为10MHz,则总线带宽是\blank{}MB/s
	\question 若数据在存储器中以小端方式存放,则十六进制数12345678H按字节地址从小到大依次为\blank{}
\end{questions}
\end{document}