%!TEX program = xelatex
\documentclass{exam}
\usepackage{ctex}
\usepackage{CJKnumb}
\newif\ifprint
%\printtrue %选择题答案/printfalse时打印,/printtrue时隐藏
\printfalse
\usepackage{ulem}
\newcommand{\blank}[2][1cm]{\uline{\makebox[#1][c]{%
			\ifprint
			\phantom{#2}%
			\else
			#2%
			\fi}}}
\pagestyle{headandfoot}

\begin{document}
\section{选择题}
\begin{questions}
	\question 冯·诺依曼计算机中指令和数据均以二进制形式存放在存储器中,CPU区分它们的依据是\blank{}
	\begin{choices}
		\choice 指令操作码的译码结果
		\choice 指令和数据的寻址方式
		\choice 指令周期的不同阶段
		\choice 指令和数据所在的存储单元
	\end{choices}

	\question 以下关于计算机历史发展的叙述,错误的是\blank{}
	\begin{choices}
		\choice 第一台电子计算机采用的是电子管技术
		\choice 早期的电子计算机,主要用于个人业务处理
		\choice 集成电路技术的发展,使电子计算机在体积和速度上有了很大改善
		\choice 摩尔定律说明了半导体集成度发展的规律
	\end{choices}

	\question 以下哪一项,不属于CPU的组成部分\blank{}
	\begin{choices}
		\choice 运算器
		\choice 控制器
		\choice 寄存器
		\choice I/O系统
	\end{choices}

	\question 下列关于总线仲裁方式的说法中,错误的为\blank{}
	\begin{choices}
		\choice 独立请求方式响应时间最快,是以增加处理机开销和增加控制线数为代价的
		\choice 计数器定时查询方式下,有一根总线请求(BR)和一根设备地址线,计数器可以从0开始增加,或接上次计数增加
		\choice 链式查询方式对电路故障最敏感
		\choice 分布式仲裁控制逻辑分散在各总线各部件中,不需要中央仲裁器
	\end{choices}

	\question 冯·诺依曼机可以区分指令和数据的部件是\blank{}
	\begin{choices}
		\choice 总线
		\choice 控制器
		\choice 控制存储器
		\choice 运算器
	\end{choices}

	\question 计算机的Cache——主存层次,主要是为了解决什么问题\blank{}
	\begin{choices}
		\choice 速度匹配问题
		\choice 存储器容量问题
		\choice 数据格式兼容问题
		\choice 电平匹配问题
	\end{choices}

	\question 总线的特征不包括以下哪一项\blank{}
	\begin{choices}
		\choice 物理特征
		\choice 功能特征
		\choice 电气特征
		\choice 时间特征
	\end{choices}

	\question 设CPU与I/O设备以中断方式进行数据传送,CPU响应中断时,该I/O设备接口控制器发送给CPU的指令中断向量表(中断向量表中放中断向量)的指针是\verb|0008H|。\verb|0008H|单元中的值是\verb|1200H|。则该I/O设备的中断服务程序在内存中的入口地址为\blank{}
	\begin{choices}
		\choice \verb|0008H|
		\choice \verb|0009H|
		\choice \verb|1200H|
        \choice \verb|1201H|
	\end{choices}
    
    \question DMA方式的接口电路中有程序中断部件,其作用为\blank{}
    \begin{choices}
    	\choice 进行预处理
    	\choice 向CPU提出总线使用权
    	\choice 向CPU提出传输结束
        \choice 检查数据是否出错
    \end{choices}

    \question 设浮点数的基数为8,尾数用原码表示,则以下\blank{}是规格化小数
    \begin{choices}
    	\choice 1.000101
    	\choice 0.000101
    	\choice 1.011011
    	\choice 0.000010
    \end{choices}
\end{questions}
\end{document}